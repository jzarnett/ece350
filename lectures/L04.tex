\documentclass[letterpaper,10pt]{article}

\usepackage{enumitem}
\usepackage{titling}
\usepackage{listings}
\usepackage{url}
\usepackage{hyperref}
\usepackage{setspace}
\usepackage{subfig}
\usepackage{sectsty}
\usepackage{pdfpages}
\usepackage{colortbl}
\usepackage{multirow}
\usepackage{multicol}
\usepackage{relsize}
\usepackage{amsmath}
\usepackage{wasysym}
\usepackage{fancyvrb}
\usepackage[yyyymmdd]{datetime}
\usepackage{amsmath,amssymb,amsthm,graphicx,xspace}
\usepackage[titlenotnumbered,noend,noline]{algorithm2e}
\usepackage[compact]{titlesec}
\usepackage{XCharter}
\usepackage[T1]{fontenc}
\usepackage[scaled]{beramono}
\usepackage[normalem]{ulem}
\usepackage{booktabs}
\usepackage{tikz}
\usetikzlibrary{arrows,automata,shapes,trees,matrix,chains,scopes,positioning,calc}
\tikzstyle{block} = [rectangle, draw, fill=blue!20,
text width=2.5em, text centered, rounded corners, minimum height=2em]
\tikzstyle{bw} = [rectangle, draw, fill=blue!20,
text width=4em, text centered, rounded corners, minimum height=2em]

\definecolor{namerow}{cmyk}{.40,.40,.40,.40}
\definecolor{namecol}{cmyk}{.40,.40,.40,.40}
\renewcommand{\dateseparator}{-}

\let\LaTeXtitle\title
\renewcommand{\title}[1]{\LaTeXtitle{\textsf{#1}}}

\lstset{basicstyle=\footnotesize\ttfamily,breaklines=true}

\newcommand{\handout}[5]{
	\noindent
	\begin{center}
		\framebox{
			\vbox{
				\hbox to 5.78in { {\bf ECE 350: Real-Time Operating Systems } \hfill #2 }
				\vspace{4mm}
				\hbox to 5.78in { {\Large \hfill #4  \hfill} }
				\vspace{2mm}
				\hbox to 5.78in { {\em #3 \hfill \today} }
			}
		}
	\end{center}
	\vspace*{4mm}
}

\newcommand{\lecture}[3]{\handout{#1}{#2}{#3}{Lecture#1}}
\newcommand{\tuple}[1]{\ensuremath{\left\langle #1 \right\rangle}\xspace}

\newcommand{\Rplus}{\protect\hspace{-.1em}\protect\raisebox{.35ex}{\smaller{\smaller\textbf{+}}}}
\newcommand{\Cpp}{\mbox{C\Rplus\Rplus}\xspace}


\addtolength{\oddsidemargin}{-1.000in}
\addtolength{\evensidemargin}{-0.500in}
\addtolength{\textwidth}{2.0in}
\addtolength{\topmargin}{-1.000in}
\addtolength{\textheight}{1.75in}
\addtolength{\parskip}{\baselineskip}
\setlength{\parindent}{0in}
\renewcommand{\baselinestretch}{1.5}
\newcommand{\term}{Spring 2023}
\newcommand{\termnumeric}{1235}

\singlespace


\begin{document}

\lecture{ 4 --- Concurrency Control Implementation}{\term}{Jeff Zarnett}


\section*{Concurrency Control Construct Implementation}
Previous discussion covered ideas about how concurrency control might actually be achieved. There were some solutions that did not work, but where we landed was that for the mutex (binary semaphore) we could use the test-and-set instruction. Remember this?

\paragraph{Test-and-Set.}
The Test-and-Set instruction is a special machine instruction that is performed in a single instruction cycle and is therefore not interruptible. It is therefore an atomic read and write. The idea is that the Test-and-Set instruction returns a boolean value. When run, it will examine the flag variable (in this example, \texttt{i}) and if it is zero, it will set it to 1 and return true. If \texttt{i} is currently set to 1, it will return false. The meaning of the return value is clear: if it is true, it is the current thread's turn to enter the critical section. The Test-and-Set instruction is not actually implemented like this, but a description of its functionality in C is:

\begin{lstlisting}[language=C]
boolean test_and_set( int* i ) {
  if ( *i == 0 ) {
    *i = 1;
    return true;
  } else {
    return false;
  }
}
\end{lstlisting}


Now, to make use of the \texttt{test\_and\_set} routine.

\begin{lstlisting}[language=C]
while ( !test_and_set( busy ) ) {
   /* Wait for my turn */
}
/* critical section */
busy = 0;
\end{lstlisting}

\paragraph{Compare-and-Swap.} That works okay for the mutex scenario, but doesn't work for the general, counting, semaphore where the values can be things other than 0 or 1 (locked and unlocked). For that we have the compare-and-swap, or sometimes it's called compare-and-exchange, instruction. Like test-and-set, it is implemented as a hardware instruction that is completed in one cycle and is uninterruptible. As before, it is not implemented like this, but a precise C-language definition looks like this:

\begin{lstlisting}[language=C]
int compare_and_swap( int * value, int old_value, int new_value ) {
  if ( *value == old_value ) {
    *value = new_value;
  }
  return *value;
}
\end{lstlisting}

And to make use of it in trying to decrement a semaphore:

\begin{lstlisting}[language=C]
int old = 1;
while (true) {
  int actual = compare_and_swap( sem, old, old - 1 );
  if ( actual == old ) {
    old = old - 1;
    break;
  } else {
    old = actual;
  }
}
/* critical section */
while (true) {
  int actual = compare_and_swap( sem, old, old + 1 );
  if ( actual == old ) {
    break;
  } else {
    old = actual;
  }
}
\end{lstlisting}

The CAS routine will change the value of the integer \texttt{value} from \texttt{old\_value} to \texttt{new\_value} if it succeeds and make no change if it did not succeed. Why might it fail? It might fail if some other thread has modified the value in the meantime. That's why we get the actual value back as the return statement of the function: so we can update the old value in the current thread. If we wanted to change it from 1 to 0 and we find it's already 0, then we just need to try again changing it from 0 to -1. That might also fail, but we will eventually succeed, even if it takes an arbitrary amount of time. It might be possible for a thread to be so unlucky it never gets a turn, but let's just say that this does not happen. And we could prevent that risk entirely if we use an alternative approach below.

The initial guess for \texttt{old} can certainly be wrong; the initial attempt to set it will fail and we'll get the correct value. Take note also the need to do the compare-and-swap operation to increment the value as well -- otherwise we'll have race conditions there.

Alternative: if there is appropriate hardware support, you could use a simpler version of this where you just attempt to do an atomic increment or atomic decrement. That prevents the scenario where multiple attempts are necessary to get the value. Just make sure to choose the correct atomic primitive that returns the \textit{new} value, since that will determine whether the calling thread should get blocked or not.

Okay, but where we left things off in ECE~252 was that sometimes we don't get the result that we want. Initial attempts and what we see here involve busy-waiting, but then we said it's better if the operating system blocks the process. That sort of hand-waving really just put off handling this scenario to later, except later is now.

\subsection*{Blocking and Unblocking}

So the caller has tried to lock the mutex or wait on the semaphore; each of those was a system call and now the operating system is running and it uses one of the methods above -- test-and-set, compare-and-swap, atomic operation -- to do the assessment of whether the thread can enter. But that happened in the kernel code.

Given the outcome of the attempt to enter the critical section, the kernel needs to take action based on that. If the caller used a try-lock or try-wait sort of call, then they don't get blocked no matter the outcome; simply send back a response that says ``no''. Then the rest of the discussion in this section does not apply.

If the thread is supposed to be blocked, that's exactly what the operating system needs to do. Change the status of the calling thread to be blocked, and then choose another thread to continue execution. Which one? Scheduling decision, as you know, except scheduling is \textit{also} something that we hand-waved away in the past and are about to find ourselves forced to revisit. Anyway, back to blocked threads.

We will need some way of knowing that this thread is blocked on the particular semaphore or mutex -- hence, an implementation of blocking and unblocking will require perhaps that the concurrency control construct has an associated queue of the threads waiting for it. Or maybe when a thread is blocked there's somewhere in its PCB that says what it's waiting for. 

Eventually, whichever thread did lock the mutex will want to unlock it or some thread will post on a semaphore. If it's a counting semaphore, we always increase its internal counter by 1 and will unblock a waiting thread. For the mutex, we'll only change the internal counter from 0 to 1 if there is no thread waiting; otherwise the only thing to do is unblock a waiting thread.

That particular implementation is not the only way to go about it, but it's important not to just set the mutex back to unlocked and forget about whatever threads are waiting -- or worse, set it back to unlocked AND unblock one of the threads and allow it to proceed. Either one of those outcomes would ruin the mutual exclusion behaviour we want.

This prompts immediately the question of what thread should be woken up when an unlock or post event occurs. Again, that's a scheduling decision, but you could choose a simple and ``fair'' approach of just taking the first thread in the queue is the one that's chosen. This might not be optimal because it ignores things like thread priorities, but it does work. 

When a thread is unblocked after waiting for the mutex or semaphore, it resumes its execution and will proceed as expected. So it turns out, the actual implementation of blocking and unblocking to have concurrency control is rather straightforward. 

\subsection*{Condition Variables}
Condition variables have a lot of similarities but need no such internal counter or associated logic.




\bibliographystyle{alphaurl}
\bibliography{350}


\end{document}